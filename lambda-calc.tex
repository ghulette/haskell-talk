%!TEX root=fp-intro.tex

\section{The $\lambda$-Calculus}

%
\begin{frame}{The untyped $\lambda$-calculus}

The $\lambda$-calculus is the formalism that underlies a functional approach to
programming.

We will start our discussion with the \textbf{untyped $\lambda$-calculus}, or
simply $\mathbf{\lambda}$ -- a Turing-complete model of computation with a
surprisingly concise description.

\end{frame}

%
\begin{frame}{The $\lambda$-calculus}

The grammar for $\lambda$:

\begin{tabular}{lll}
  $t$ \texttt{::=} & $x$              & \emph{(variable)} \\
                   & $t_1 \; t_2$     & \emph{(application)} \\
                   & $\lambda x . t$  & \emph{(abstraction)} \\
\end{tabular}

\end{frame}

%
\begin{frame}{Computation = evaluation}

What do we mean by semantics?

\end{frame}

%
\begin{frame}{Call-by-value}



\end{frame}

%
\begin{frame}{Call-by-need}

This is more-or-less the semantics used by Haskell.

\end{frame}

%
\begin{frame}{Scope}

Dynamic vs. static

\end{frame}

%
\begin{frame}{Recursion}

Y-combinator, example.

\end{frame}

%
\begin{frame}{Types}

Typing is an alternative way of evaluating an expression. Present the
simply-typed $\lambda$-calculus, note the problems.

\end{frame}

