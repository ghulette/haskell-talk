%!TEX root=fp-intro.tex

\section{The $\lambda$-Calculus}

%
\begin{frame}{The untyped $\lambda$-calculus}

The \textbf{untyped $\lambda$-calculus} is the formalism that underlies
functional programming.

It is a \textbf{core language} for many functional languages. Convenient
language structures can be defined by translation into $\lambda$-calculus.

Haskell actually uses a typed variant called \textbf{System F}, but the
principles are the same.

\end{frame}

%
\begin{frame}{The $\lambda$-calculus}

The grammar is extremely simple.

\begin{block}{Grammar for the simple untyped $\lambda$-calculus}
\begin{tabular}{lll}
  $e$ \texttt{::=} & $x$              & \emph{(variable)} \\
                   & $e_1 \; e_2$     & \emph{(application)} \\
                   & $\lambda x . e$  & \emph{(abstraction)} \\
\end{tabular}
\end{block}

Parentheses disambiguate expressions.

\end{frame}

%
\begin{frame}{Semantics}

The only means to ``compute'' in this language is to apply functions to
arguments (which are themselves functions). For example, we can define the
identity function, which just returns its argument:

\begin{block}{Identity function}
\[
\mathtt{id} = \lambda x . x
\]
\end{block}

\end{frame}

%
\begin{frame}{Semantics}

Given an expression in the $\lambda$-calculus, we can \textbf{evaluate} it by
manipulating expressions according to the rules of some \textbf{operational
semantics}.

The most popular operational semantics are \textbf{call-by-value}.

Haskell's semantics are based on a different semantics -- \textbf{call-by-need}.

\end{frame}

%
\begin{frame}{Notation}

Operational semantics have their own idiosyncratic notation.

\begin{columns}[t]
\begin{column}{0.5\textwidth}

\begin{block}{Conditions}
Read $\infer{c}{a \; b}$ as ``if $a$ and $b$, then $c$''.
\end{block}

\begin{block}{Transitions}
Read $a \to b$ as ``$a$ rewrites to $b$''.
\end{block}

\end{column}
\begin{column}{0.4\textwidth}

\begin{block}{Renaming}
Read $[x \mapsto a] b$ as ``$x$ is rewritten to $a$ wherever $x$ occurs in the 
expression $b$''.
\end{block}

\end{column}
\end{columns}

\end{frame}


%
\begin{frame}{Call-by-value}

\begin{columns}
\begin{column}{0.4\textwidth}

\begin{block}{Expressions}
\begin{tabular}{ll}
$e$ \texttt{::=} & $x$              \\
                 & $e \; e$         \\
                 & $\lambda x . e$  \\
\end{tabular}
\end{block}

\begin{block}{Values}
\begin{tabular}{ll}
$v$ \texttt{::=} & $\lambda x . e$  \\
\end{tabular}
\end{block}

\end{column}
\begin{column}{0.4\textwidth}

\begin{block}{Evaluation}
\[
\infer{e_1 e_2 \to e_1' e_2}{e_1 \to e_1'}
\]
\[
\infer{v_1 e_2 \to v_1' e_2}{e_2 \to e_2'}
\]
\[
\infer{(\lambda x . e) v \to \left[x \mapsto v \right] \, e}{}
\]
\end{block}

\end{column}
\end{columns}

The key thing to understand is that call-by-value forces arguments to be reduced
\textbf{before} a function is called.

\end{frame}

%
\begin{frame}{Call-by-need}

\begin{columns}
\begin{column}{0.4\textwidth}

\begin{block}{Expressions}
\begin{tabular}{ll}
$e$ \texttt{::=} & $x$              \\
                 & $e \; e$         \\
                 & $\lambda x . e$  \\
\end{tabular}
\end{block}

\begin{block}{Values}
\begin{tabular}{ll}
$v$ \texttt{::=} & $\lambda x . e$  \\
\end{tabular}
\end{block}

\end{column}
\begin{column}{0.4\textwidth}

\begin{block}{Evaluation}
\[
\infer{e_1 e_2 \to e_1' e_2}{e_1 \to e_1'}
\]
\[
\infer{(\lambda x . e_1) e_2 \to \left[x \mapsto e_2 \right] \, e_1}{}
\]
\end{block}

\end{column}
\end{columns}

In call-by-need, arguments are not reduced until they are needed (i.e., when
they are applied).

\end{frame}

%
\begin{frame}{Extensions}

We can extend the untyped $\lambda$-calculus with natural numbers and simple
arithmetic -- this makes examples easier to grok.

We also introduce a \texttt{let} syntactic sugar:

\begin{block}{\texttt{let} expressions}
Define

\[
\mathtt{let} \; x \; \mathtt{=} \; e_1 \; \mathtt{in} \; e_2
\]

to mean

\[
(\lambda x . e_2) \; e_1
\]
\end{block}

\end{frame}

%
\begin{frame}[fragile]{Syntax}

In the following examples I use Haskell's notation for $\lambda$ expressions:

\begin{block}{Expressions}
\begin{tabular}{ll}
$\lambda$       & Haskell \\
\hline
$x$             & \texttt{x} \\
$e_1 e_2$       & \texttt{e1} \texttt{e2} \\
$\lambda x . e$ & \verb!\x -> e! \\
\end{tabular}
\end{block}

\end{frame}

%
\begin{frame}[fragile]{Example}

\begin{block}{Example}
\begin{verbatim}
let f = \x -> x + x in f (1 + 2)
\end{verbatim}
\end{block}

This expression 

\begin{itemize}
  \item defines a function $f$ which adds its argument to itself;
  \item applies $f$ to the expression $1+2$.
\end{itemize}

\end{frame}

%
\begin{frame}[fragile]{Example}

\begin{block}{Evaluation (call-by-value)}
\begin{verbatim}
   (\f -> f (1 + 2)) (\x -> x + x)
=> (\x -> x + x) (1 + 2)
=> (\x -> x + x) 3
=> 3 + 3
=> 6
\end{verbatim}
\end{block}

\begin{block}{Evaluation (call-by-need)}
\begin{verbatim}
   (\f -> f (1 + 2)) (\x -> x + x)
=> (\x -> x + x) (1 + 2)
=> (1 + 2) + (1 + 2)
=> ...
=> 6
\end{verbatim}
\end{block}

\end{frame}

%
\begin{frame}{Recursion}

It may seem like complicated control flow would be difficult or impossible in
the untyped $\lambda$-calculus.

Not true! We can use \textbf{fixpoint combinators} to define general recursive
functions. 

In fact, the untyped $\lambda$-calculus is \textbf{Turing complete}!

Recall that a \textbf{fixed point} of a function $f$ is a value $x$ such that
$f(x) = x$.

\end{frame}

%
\begin{frame}{Y-combinator}

The \textbf{Y-combinator} is a famous fixpoint function.

\begin{block}{Y-combinator}
$Y = \lambda f.(\lambda x.f (x \; x)) (\lambda x.f (x \; x))$
\end{block}

In fact, there are infinitely many such combinators.

\end{frame}

%
\begin{frame}[fragile]{Recursion}

We can use a fixpoint combinator to essentially generate new copies of a
function body, as needed. This gives us recursion.

\begin{block}{Factorial function}
\begin{verbatim}
let fix = \f -> (\x -> f (x x)) (\x -> f (x x)) in 
let fact = 
  fix (\g -> (\n -> if n = 0 then 1 else n * (g (n - 1)))) 
in
fact 5
-- fact 5 => 120
\end{verbatim}
\end{block}

Just remember that this gives us a formal basis for recursion.

\end{frame}
