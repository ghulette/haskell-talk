%!TEX root=fp-intro.tex

\section{Types}

%
\begin{frame}{Basic types in Haskell}

We just saw the \emph{untyped} $\lambda$-calculus, which does not have types.

There are a number of ways to augment the $\lambda$-calculus with types.
Haskell's types are based on a calculus called \textbf{System F}.

\end{frame}

%
\begin{frame}{Features}

Haskell's type system is 

\begin{itemize}
  \item \textbf{strong},
  \item \textbf{static},
  \item and \textbf{inferred}.
\end{itemize}

\end{frame}

%
\begin{frame}[fragile]{Strong typing}

\textbf{Strong typing} is a kind of lightweight formal method. It guarantees
that certain things can never go wrong in a Haskell program.

\begin{block}{Strong typing}
\begin{verbatim}
> 1 + True
Error: No instance for (Num Bool)
\end{verbatim}
\end{block}

\end{frame}

%
\begin{frame}[fragile]{Static typing}

\textbf{Static typing} means that programs which violate typing rules are
rejected at compile time (as opposed to runtime).

This can have benefits for performance, among other things.

\begin{block}{What are types?}
\begin{itemize}
  \item Regular semantics evaluate a program to produce an answer.
  \item Typing semantics evaluate a program to flag errors.
\end{itemize}

For static types, we prefer that the evaluation always terminates in a
reasonable amount of time. This implies that the typing semantics cannot be
Turing complete. 
\end{block}

\end{frame}

%
\begin{frame}[fragile]{Type inference}

Haskell can, in many cases, figure out types for you. This is very convenient.

You can always declare types yourself, for any expression or sub-expression.
Haskell will check your annotations against its own inference.

It can even figure out the most general polymorphic type (often more general
than you would think).

\begin{block}{Type inference}
\begin{verbatim}
> :t \x -> x * 2
\x -> x * 2 :: Num a => a -> a
\end{verbatim}
\end{block}

\end{frame}

%
\begin{frame}[fragile]{Basic types in Haskell}

We write $e::t$ to say that an expression $e$ has type $t$.

\begin{block}{}
\begin{verbatim}
> :set +t  -- Tell ghci to print types
> 'a'
'a'
it :: Char
> 2^10
1024
it :: Integer
> 5 == (2+3)
True
it :: Bool
\end{verbatim}
\end{block}

\textbf{Note}: in \texttt{ghci}, ``\texttt{it}'' is the value of the last
expression.

\textbf{Note}: \texttt{Int} is the machine's native word length.
\texttt{Integer} is unbounded in size (and more expensive).

\end{frame}

%
\begin{frame}[fragile]{Tuples}

Haskell supports \textbf{product types} with \textbf{fixed} size:

\begin{block}{}
\begin{verbatim}
> (2,3)
(2,3)
it :: (Integer, Integer)
> (False,5,'a')
(False,5,'a')
it :: (Bool, Integer, Char)
\end{verbatim}
\end{block}

\end{frame}

%
\begin{frame}[fragile]{Lists}

Lists are a fundamental datatype in functional programming.

\begin{block}{}
\begin{verbatim}
> ['a','b','c']
['a','b','c']
it :: [Char]
> [1..10]
[1,2,3,4,5,6,7,8,9,10]
it :: [Integer]
\end{verbatim}
\end{block}

\end{frame}

%
\begin{frame}[fragile]{Strings}
  
There is a \texttt{String} type -- it is just a synonym for \texttt{[Char]}.

\begin{block}{}
\begin{verbatim}
> "Hello world"
"Hello world"
it :: [Char]
\end{verbatim}
\end{block}

\end{frame}

%
\begin{frame}[fragile]{Functions}
  
Functions are typed, just like anything else.

\begin{block}{}
\begin{verbatim}
> :t \x -> x + 1
\x -> x + 1 :: Int -> Int
\end{verbatim}
\end{block}

We read this as ``Int to Int,'' i.e. the function takes an int as input and
produces an int as output.

\end{frame}

%
\begin{frame}[fragile]{Algebraic Data Types}

What is a \texttt{Bool}?

\begin{block}{}
\begin{verbatim}
data Bool = False | True
\end{verbatim}
\end{block}

Or, similarly,

\begin{block}{}
\begin{verbatim}
data Color = Red | Green | Blue
\end{verbatim}
\end{block}

These are called \textbf{algebraic data types}.

The type name is given on the left-hand side. The \textbf{constructor(s)}
are on the right.

\end{frame}

%
\begin{frame}[fragile]{Algebraic Data Types}

In Haskell, lists are just another data type.

\begin{block}{Define \texttt{IntList}}
\begin{verbatim}
data IntList = Nil
             | Cons Int IntList
\end{verbatim}
\end{block}

Notice how \texttt{IntList} is defined recursively.

\begin{block}{Build some lists}
\begin{verbatim}
> lst = Cons 1 (Cons 2 (Cons 3 Nil))
lst :: IntList
> lst' = 1 `Cons` 2 `Cons` 3 `Cons` Nil
lst' :: IntList
\end{verbatim}
\end{block}

\end{frame}

%
\begin{frame}[fragile]{Algebraic Data Types}

Regular Haskell lists are defined in the same way.

\begin{block}{}
\begin{verbatim}
> 1 : 2 : 3 : []
[1,2,3] :: [Int]
\end{verbatim}
\end{block}

The bracket syntax is just syntactic sugar.

\end{frame}

%
\begin{frame}[fragile]{Tuples}

We can define fixed-size tuples in a similar way:

\begin{block}{}
\begin{verbatim}
data Tuple2 a b = Tuple2 a b
data Tuple3 a b c = Tuple3 a b c

x = Tuple3 True 5 "Hello"
-- x :: Tuple3 Bool Integer String
\end{verbatim}
\end{block}

Haskell has very few built-in data types. Most are defined algebraically.

\end{frame}
