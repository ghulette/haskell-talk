%!TEX root=fp-intro.tex

\section{First-order functions}

%
\begin{frame}[fragile]{Simple functions}

Functions in Haskell are abstractions in the $\lambda$-calculus. 

Think of functions as expressions with missing bits, which are represented by
the inputs.

\end{frame}

%
\begin{frame}[fragile]{Example}

Here is a very simple Haskell function which doubles a number:

\begin{block}{}
\begin{verbatim}
double x = 2 * x
\end{verbatim}
\end{block}

Or, equivalently:

\begin{block}{}
\begin{verbatim}
double = \x -> 2 * x
\end{verbatim}
\end{block}

\end{frame}

%
\begin{frame}[fragile]{Example}

We can (and often do) provide an explicit type for top-level functions:

\begin{block}{}
\begin{verbatim}
double :: Int -> Int
double x = 2 * x
\end{verbatim}
\end{block}

Haskell will infer the type anyway, and check that it matches your annotation.

\end{frame}

%
\begin{frame}[fragile]{Anonymous functions}

Functions can be defined and used just like any other value.

\begin{block}{}
\begin{verbatim}
> let y = 1 in 
  let foo = \x -> x + y in 
  foo 5
6
\end{verbatim}
\end{block}

\end{frame}

%
\begin{frame}[fragile]{Multiple arguments}

In the $\lambda$-calculus and Haskell, functions \textbf{always have one
argument}.

There are two different ways to simulate multiple arguments. The first style is
similar to that of imperative languages -- pack the arguments into a data
structure (e.g., a tuple):

\begin{block}{}
\begin{verbatim}
> let add = \(x,y) -> x + y
add :: (Integer, Integer) -> Integer
> add (5,2)
7
\end{verbatim}
\end{block}

\end{frame}

%
\begin{frame}[fragile]{Multiple arguments}

But in Haskell and most functional languages, the preferred style is
\textbf{curried}:

\begin{block}{}
\begin{verbatim}
> let add' = \x -> \y -> x + y
add' :: Integer -> Integer -> Integer
> add' 5 2
7
\end{verbatim}
\end{block}

\end{frame}

%
\begin{frame}[fragile]{Multiple arguments}

These all mean the same thing:

\begin{block}{}
\begin{verbatim}
> let add' = \x -> \y -> x + y
add' :: Integer -> Integer -> Integer
\end{verbatim}
\end{block}

\begin{block}{}
\begin{verbatim}
> let add'' = \x y -> x + y
add'' :: Integer -> Integer -> Integer
\end{verbatim}
\end{block}

\begin{block}{}
\begin{verbatim}
> let add''' x y = x + y
add''' :: Integer -> Integer -> Integer
\end{verbatim}
\end{block}

\end{frame}

%
\begin{frame}[fragile]{Multiple arguments}

Curried-style functions apply the first argument to the outermost function, and
return a new function.

\begin{block}{}
\begin{verbatim}
> let add' = \x -> (\y -> x + y)
add' :: Integer -> (Integer -> Integer)
> :t (add' 5)
add' 5 :: Integer -> Integer
> :t (add' 5) 2
add' 5 2 :: Integer
\end{verbatim}
\end{block}

In other words, \texttt{add'} looks like a function that takes two arguments,
but in fact it is a function that takes one argument, and returns another
function that takes one argument.

\end{frame}

%
\begin{frame}[fragile]{Multiple arguments}

What does \texttt{add2 5} mean, exactly?

\begin{block}{}
\begin{verbatim}
   add2 5
=> (\x -> \y -> x + y) 5  -- Definition of add2
=> \y -> 5 + y            -- Application
\end{verbatim}
\end{block}

This is one of the best features of Haskell: we can always substitute equals for
equals to determine the meaning of an expression.

\end{frame}


%
\begin{frame}[fragile]{Partial application}

\textbf{Partial application} can be quite useful for building specialized
functions out of general ones.

\begin{block}{ghci}
\begin{verbatim}
> let succ = add' 1
> succ 3
4
\end{verbatim}
\end{block}

Reordering inputs is easy too, if necessary.

\end{frame}

%
\begin{frame}[fragile]{Recursion}

Recursion is very easy in Haskell. Definitions exist in their own scope.

\begin{block}{}
\begin{verbatim}
> fib n = if n <= 1 then 1 else fib (n - 1) + fib (n - 2)
fib :: Int -> Int
> fib 5
8
\end{verbatim}
\end{block}

\end{frame}

%
\begin{frame}[fragile]{Pattern matching}

Algebraic datatypes can be \textbf{pattern matched} in functions.

\begin{block}{}
\begin{verbatim}
data IntList = Nil
             | Cons Int IntList

toList :: IntList -> [Int]
toList xs = case xs of Nil -> []
                       Cons x xs' -> x : (toList xs')
\end{verbatim}
\end{block}

\end{frame}

%
\begin{frame}[fragile]{Pattern matching}

We can omit the \texttt{case} statement and match directly in the arguments:

\begin{block}{}
\begin{verbatim}
sum :: [Int] -> Int
sum [] = 0
sum (x:xs) = x + (sum xs)
\end{verbatim}
\end{block}

\begin{block}{ghci}
\begin{verbatim}
> sum [1..10]
55
\end{verbatim}
\end{block}

\end{frame}

%
\begin{frame}[fragile]{Pattern matching}

\begin{block}{}
\begin{verbatim}
data Tree a = Tip | Node a (Tree a) (Tree a)
empty = Tip
leaf x = Node x Tip Tip

insert :: Int -> Tree Int -> Tree Int
insert x Tip = leaf x
insert x (Node y l r) | x < y     = Node y (insert x l) r
                      | otherwise = Node y l (insert x r)

height Tip = 0
height (Node _ l r) = 1 + max (height l) (height r)
\end{verbatim}
\end{block}

\begin{block}{ghci}
\begin{verbatim}
> let t = insert 5 (insert 25 (insert 15 (insert 20 empty)))
> height t
3
\end{verbatim}
\end{block}

\end{frame}

%
\begin{frame}[fragile]{Non-strict evaluation}

We can do some interesting things with non-strict evaluation.

\begin{block}{Defining $\bot$}
\begin{verbatim}
> let bot = bot
bot :: t
> bot
^C
\end{verbatim}
\end{block}

Evaluating \texttt{bot} will cause the Haskell runtime to enter an infinite
loop.

\end{frame}

%
\begin{frame}[fragile]{Non-strict evaluation}

In Haskell, we can actually work with values like this.

\begin{block}{Working with $\bot$}
\begin{verbatim}
> let snd = \x y -> y
> snd bot 5
5
\end{verbatim}
\end{block}

\end{frame}

%
\begin{frame}[fragile]{Non-strict evaluation}

A less useless example:

\begin{block}{Infinite Fibonacci}
\begin{verbatim}
> let fibs = 0 : 1 : zipWith (+) fibs (tail fibs)
fibs :: [Integer]
> take 10 fibs
[0,1,1,2,3,5,8,13,21,34]
> fibs
[0,1,1,2,3,5,8,13,21,34,55,89,144,233,377,610,987,1597,2584,
4181,6765,10946,17711,28657,46368^C
\end{verbatim}
\end{block}

In the last line, \texttt{ghci} will try to evaluate the (infinite) list of
Fibonacci numbers forever.

\end{frame}
