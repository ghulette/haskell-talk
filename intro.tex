%!TEX root = fp-intro.tex

%
\begin{frame}{Motivation}

In this talk I will introduce \textbf{functional programming}, in the context of
the \textbf{Haskell} programming language.

It is intended to provide a short ``highlight reel'' of how functional
programming works and how it is different from other languages you may be used
to.

The goal is to support my research group's interest in \textbf{formal methods}.

\end{frame}

%
\begin{frame}{Why functional programming?}

Focus is on \textbf{combining small functions} in interesting ways.

In a \textbf{pure} functional language like Haskell, computation is the process
by which we reduce an expression to a normal form.

\begin{itemize}
  \item Avoids statefulness and mutable data;
  \item Functions depend only on their apparent inputs;
  \item Strong typing eliminates many common errors.
\end{itemize}

\end{frame}

%
\begin{frame}[fragile]{Why functional programming?}

Pure functional programs are \textbf{referentially transparent}.

This means that an expression can always be replaced by its value, without
changing the meaning of the program.

Consider the following imperative program:

\begin{block}{}
\begin{verbatim}
int foo() {
  print "hello"
  return 5
}

x = foo()
print (x + x)
\end{verbatim}
\end{block}

\end{frame}

%
\begin{frame}[fragile]{Why functional programming?}

It is easier to reason about pure functional programs because you can
\textbf{substitute equals for equals}.

Similar to how we do proofs in mathematics.

\end{frame}

%
\begin{frame}{Why Haskell?}

Haskell isn't the only functional language, but it is a good one.

\begin{itemize}
  \item Clean, concise syntax;
  \item Strong, static typing;
  \item Flexible polymorphism;
  \item Mature compiler, tools, and libraries;
  \item Friendly, helpful community.
\end{itemize}

I use Haskell for 80\% of programming tasks (the rest is mostly C).

\end{frame}

\begin{frame}
  \frametitle{Outline}
  \tableofcontents
\end{frame}
