%!TEX root = fp-intro.tex

%
\begin{frame}{Motivation}

You should approach this talk as a tutorial on functional programming, which
happens to take the form of a tutorial on Haskell.

In particular, I want to focus on the formal structure and semantics of
functional programs, which makes them amenable to analysis.

\end{frame}

%
\begin{frame}{Why functional programming?}

Focus is on \textbf{combining simple functions} in interesting ways.

In a pure functional language like Haskell, computation is the process by which
we reduce an expression to a normal form.

\begin{itemize}
  \item Avoids state and mutable data;
  \item Functions depend only on their apparent inputs;
  \item Strong typing eliminates many common errors.
\end{itemize}

Haskell has a clever way to incorporate side effects and state.

\end{frame}

%
\begin{frame}[fragile]{Why functional programming?}

Pure functional programs are \textbf{referentially transparent}.

This means that an expression can always be replaced by its value, without
changing the meaning of the program.

Consider the following imperative program:

\begin{block}{}
\begin{verbatim}
int foo() {
  print "hello"
  return 5
}

x = foo()
print (x + x)
\end{verbatim}
\end{block}

\end{frame}

%
\begin{frame}[fragile]{Why functional programming?}

It is easier to reason about functional programs because in most cases you can
\textbf{substitute equals for equals}, just like we do in mathematics.

\end{frame}

%
\begin{frame}{Why Haskell?}

Haskell isn't the only functional language, but it is a good one.

\begin{itemize}
  \item Clean and concise syntax;
  \item Strong, static typing;
  \item Flexible polymorphism;
  \item Mature compiler and tools;
  \item Friendly, helpful community.
\end{itemize}

I use Haskell for 90\% of my programming tasks (the rest is mostly C).

\end{frame}

\begin{frame}
  \frametitle{Outline}
  \tableofcontents
\end{frame}

% \begin{frame}{Monads}
% 
% Haskell has a clever way of allowing side effects and state, while maintaining
% purity and non-strict semantics. It isolates these effects in a structure
% called a \textbf{monad}.
% 
% Monads are ubiquitous in Haskell. They are like little imperative languages
% with an interpreter, embedded in our pure functional language.
% 
% There is one ``magic'' monad, called \texttt{IO}, which will perform side
% effects like opening and reading files.
% 
% We are not going to talk much about monads today.
% 
% \end{frame}
