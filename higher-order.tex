%!TEX root=fp-intro.tex

\section{Higher-order functions}

%
\begin{frame}[fragile]{Higher-order functions}

Haskell permits \textbf{higher-order} functions: functions that take other
functions as arguments.

Higher-order functions allow us to abstract control flow in the same way we
abstract data types.

Especially when combined with non-strict evaluation, this results in
\textbf{composable} code.

\end{frame}

%
\begin{frame}[fragile]{Function composition}

The canonical example is good old function composition.

Given two functions $f$ and $g$, we define their composition 

\[
(f \circ g) x = f(g(x))
\]

or in Haskell

\begin{block}{}
\begin{verbatim}
(.) :: (b -> c) -> (a -> b) -> (a -> c)
f . g = \x -> f (g x)
\end{verbatim}
\end{block}

\end{frame}

%
\begin{frame}[fragile]{Function composition}

\begin{block}{Examples}
\begin{verbatim}
> ((+1) . (*2)) 5
11
> ((*2) . (+1)) 5
12
> let odd' = even . (+1)
> odd' 5
True
\end{verbatim}
\end{block}

\end{frame}

%
\begin{frame}[fragile]{Mapping over a list}

Lists are good candidates for higher-order functions.

We can \textbf{map} a single function over each element of a list.

\begin{block}{}
\begin{verbatim}
> let xs = [1..10]
> map (*2) xs
[2,4,6,8,10,12,14,16,18,20]
> :t even
even :: Integral a => a -> Bool
> map even xs
[False,True,False,True,False,True,False,True,False,True]
\end{verbatim}
\end{block}

\end{frame}

%
\begin{frame}[fragile]{Implementing \texttt{map}}

What is the type of \texttt{map}?

\begin{block}{}
\begin{verbatim}
> :t map
map :: (a -> b) -> [a] -> [b]
\end{verbatim}
\end{block}

\end{frame}

%
\begin{frame}[fragile]{Implementing \texttt{map}}

Writing \texttt{map} is trivial:

\begin{block}{}
\begin{verbatim}
map _ [] = []
map f (x:xs) = (f x) : map f xs
\end{verbatim}
\end{block}

\end{frame}

%
\begin{frame}[fragile]{Folding a list}

What if we want to combine the elements of a list?

\begin{block}{}
\begin{verbatim}
> foldr (+) 0 [1..10]
55
> let prod = foldr (*) 1
prod :: [Integer] -> Integer
> prod [1..10]
3628800
\end{verbatim}
\end{block}

\end{frame}

%
\begin{frame}[fragile]{Folding a list}

Folds do not need to ``reduce'' their list argument:

\begin{block}{}
\begin{verbatim}
> let echoes = foldr (\x xs -> (replicate x x) ++ xs) []
> echoes [1..10]
[1,2,2,3,3,3,4,4,4,4,5,5,5,5,5,6,6,6,6,6,6,7,7,7,7,7,7,7,
8,8,8,8,8,8,8,8,9,9,9,9,9,9,9,9,9,10,10,10,10,10,10,10,
10,10,10]
\end{verbatim}
\end{block}

\end{frame}

%
\begin{frame}[fragile]{Folding a list}

There are actually two ways to define a fold. This one moves ``right'' through
the list.

\begin{block}{}
\begin{verbatim}
foldr :: (a -> b -> b) -> b -> [a] -> b
foldr f z []     = z 
foldr f z (x:xs) = f x (foldr f z xs)
\end{verbatim}
\end{block}

\end{frame}

%
\begin{frame}[fragile]{Filtering a list}

What if we want to filter a list by some predicate?

\begin{block}{}
\begin{verbatim}
> filter even [1..10]
[2,4,6,8,10]
> filter (\x -> (x > 10) && (x < 20)) [1..100]
[11,12,13,14,15,16,17,18,19]
\end{verbatim}
\end{block}

\end{frame}

%
\begin{frame}[fragile]{Building complex functions}

We can start to compose simple functions to form more complex ones.

\begin{block}{}
\begin{verbatim}
> let maxWordLen = foldl max 0 . map length . words
maxWordLen :: String -> Int
> maxWordLen "The quick brown fox jumped"
6
\end{verbatim}
\end{block}

This is starting to look like shell programming!

\end{frame}

%
\begin{frame}[fragile]{Building complex functions}

Let's break that down.

\begin{block}{}
\begin{verbatim}
> words "The quick brown fox jumped"
["The","quick","brown","fox","jumped"]
> map length it
[3,5,5,3,6]
> foldl max 0 it
6
\end{verbatim}
\end{block}

\end{frame}

%
\begin{frame}[fragile]{Non-strict evaluation}

When we combine higher-order functions with non-strict evaluation, things get
interesting. 

Recall that \texttt{fibs} is our infinite list of Fibonacci numbers.

\begin{block}{}
\begin{verbatim}
-- First Fibonacci number greater than 10000
> (head . filter (>10000)) fibs
10946

-- First 10 even Fibonacci numbers
> (take 10 . filter even) fibs
[0,2,8,34,144,610,2584,10946,46368,196418]
\end{verbatim}
\end{block}

Consider how much control flow logic is implicit in the above examples.

\end{frame}

%
\begin{frame}[fragile]{Non-strict evaluation}

With higher-order functions and non-strict evaluation, we can write code that is
highly \textbf{composable}.

Composability is necessary for formal analysis.

\end{frame}

