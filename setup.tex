%!TEX root = fp-intro.tex

\section{Getting started with GHC}

%
\begin{frame}{Step 1: Install GHC}

GHC is written in Haskell, so you need to bootstrap it. The simplest method is
download the Haskell Platform.

The Platform includes some other helpful tools including a package manager and
some common libraries.

\end{frame}

%
\begin{frame}{Step 2: Try the interactive mode}

Like many functional languages, GHC includes both a compiler for stand-alone
programs (\texttt{ghc}) and an interactive interpreter \texttt{ghci}.

I don't actually use \texttt{ghci} very much, but it is useful sometimes and
good for testing out quick ideas.

\end{frame}

%
\begin{frame}{Step 2: Try the interactive mode}

Like many functional languages, GHC includes both a compiler for stand-alone
programs (\texttt{ghc}) and an interactive interpreter \texttt{ghci}.

I don't actually use \texttt{ghci} very much, but it is useful sometimes and
good for testing out quick ideas.

\end{frame}
