%!TEX root = fp-intro.tex

\section{Getting started with GHC}

%
\begin{frame}{GHC}

The \textbf{G}lasgow \textbf{H}askell \textbf{C}ompiler is the best compiler
currently available for Haskell.

GHC benefits from both academic research and industry support.

\end{frame}

%
\begin{frame}{Installing GHC}

GHC itself is written in Haskell. So, you need to bootstrap it.

My recommendation: \textbf{download binaries}.

The \textbf{Haskell Platform} includes GHC along with some other helpful tools,
including a package manager and useful libraries.

\center{\url{http://hackage.haskell.org/platform/}}

\end{frame}

%
\begin{frame}{Interactive mode}

GHC includes both a compiler for stand-alone programs (\texttt{ghc}) and an
interactive interpreter (\texttt{ghci}).

Useful for quickly testing out ideas or looking up types.

\emph{Confession}: I don't actually use \texttt{ghci} very often.

\end{frame}

%
\begin{frame}[fragile]{\texttt{ghci} example}

\begin{block}{}
\begin{verbatim}
$ ghci
GHCi, version 7.0.3: http://www.haskell.org/ghc/
:? for help
Prelude> print "Hello world!"
"Hello world!"
Prelude> 1+1
2
Prelude> 5 == (2+3)
True
Prelude> :q
Leaving GHCi.
\end{verbatim}
\end{block}

\end{frame}

%
\begin{frame}[fragile]{GHC: Compiling programs}

The regular compiler is \texttt{ghc}. We can take a program like this:

\begin{block}{\texttt{helloworld.hs}}
\begin{verbatim}
-- This is my program!
main = do
  putStrLn "Hello world!"
\end{verbatim}
\end{block}

and compile it using \texttt{ghc} like this:

\begin{block}{}
\begin{verbatim}
$ ghc --make helloworld.hs 
[1 of 1] Compiling Main (helloworld.hs, helloworld.o)
Linking helloworld ...
$ ./helloworld 
Hello world!
\end{verbatim}
\end{block}

\end{frame}
