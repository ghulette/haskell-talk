%!TEX root = fp-intro.tex

\section{Getting started with GHC}

%
\begin{frame}{GHC}

The \textbf{G}lasgow \textbf{H}askell \textbf{C}ompiler is the best compiler
currently available for Haskell.

GHC benefits from both academic research and industry support.

\end{frame}

%
\begin{frame}{Installing GHC}

GHC itself is written in Haskell, so you need to bootstrap it. 

The simplest solution is download binaries for your platform. Most popular
configurations are well supported.

I recommend the Haskell Platform, which includes GHC along with some other
helpful tools, including a package manager and useful libraries.

\center{\url{http://hackage.haskell.org/platform/}}

\end{frame}

%
\begin{frame}{GHC: Interactive mode}

GHC includes both a compiler for stand-alone programs (\texttt{ghc}) and an
interactive interpreter \texttt{ghci}.

I don't actually use \texttt{ghci} very much, but it is useful sometimes and
good for testing out quick ideas.

\end{frame}

%
\begin{frame}[fragile]{GHC: \texttt{ghci} example}

\begin{verbatim}
$ ghci
GHCi, version 7.0.3: http://www.haskell.org/ghc/
:? for help
Prelude> print "Hello world!"
"Hello world!"
Prelude> 1+1
2
Prelude> 5 == (2+3)
True
Prelude> :q
Leaving GHCi.
$
\end{verbatim}

\end{frame}

%
\begin{frame}[fragile]{GHC: Compiling programs}

The regular compiler is called simply \texttt{ghc}. If we have a program like
this:

\begin{verbatim}
-- helloworld.hs
-- This is my great program.

main = do
  putStrLn "Hello world!"
\end{verbatim}

then we can compile and run it like this:

\begin{verbatim}
$ ghc --make helloworld.hs 
[1 of 1] Compiling Main   ( helloworld.hs, helloworld.o )
Linking helloworld ...
$ ./helloworld 
Hello world!
\end{verbatim}

\end{frame}
