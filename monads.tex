%!TEX root=fp-intro.tex

\section{Monadic I/O}

%
\begin{frame}[fragile]{IO in Haskell}

Haskell's non-strict semantics means that IO must be handled specially.

For example, how would we construct an expression that prints two integers?

\end{frame}

%
\begin{frame}[fragile]{IO in strict languages}

Most functional languages (with strict semantics) deal with this by having an
explicit sequencing operator, something like this:

\begin{block}{}
\begin{verbatim}
-- Note: not valid Haskell!
--   print :: String -> ()
--   (;)   :: () -> () -> ()
> print "Hello" ; print "world"
Hello
world
()
\end{verbatim}
\end{block}

The type \texttt{()} is called \textbf{unit}, and it has exactly one value,
namely \texttt{()}.

Why won't this scheme work for Haskell?

\end{frame}

%
\begin{frame}[fragile]{Monads to the rescue}

To handle IO, Haskell uses a structure called a \textbf{monad}.

Monads end up having several uses in Haskell, and IO is only slightly magical.

We are going to build up to monads by way of several simpler structures.

\end{frame}

%
\begin{frame}[fragile]{Maybe}

First, let's introduce the \texttt{Maybe} data type.

\begin{block}{Maybe.hs}
\begin{verbatim}
data Maybe a = Just a
             | Nothing
             deriving (Eq,Show)
\end{verbatim}
\end{block}

\end{frame}

%
\begin{frame}[fragile]{Maybe}

\texttt{Maybe} is useful for functions with default parameters, or where things
might fail.

\begin{block}{Maybe.hs}
\begin{verbatim}
find :: (a -> Bool) -> [a] -> Maybe a
find _ [] = Nothing
find pred (x:xs) = if pred x then Just x else find p xs
\end{verbatim}
\end{block}

\begin{block}{}
\begin{verbatim}
> find (==5) [1..10]
Just 5
> find even [1..10]
Just 2
> find (>10) [1..10]
Nothing
\end{verbatim}
\end{block}

\end{frame}

%
\begin{frame}[fragile]{Functors}

Let's make a very simple typeclass called a \textbf{functor}, which will allow
us to ``map'' functions over values in a ``context.''

\begin{block}{}
\begin{verbatim}
class Functor f where
  fmap :: (a -> b) -> f a -> f b
\end{verbatim}
\end{block}

This looks just like \texttt{map} for lists!

\begin{block}{}
\begin{verbatim}
instance Functor [] where
  fmap _ [] = []
  fmap f (x:xs) = f x : fmap f xs
\end{verbatim}
\end{block}

We can also make \texttt{Maybe} a functor:

\begin{block}{}
\begin{verbatim}
instance Functor Maybe where
  fmap _ Nothing = Nothing
  fmap f (Just x) = Just (f x)
\end{verbatim}
\end{block}

\end{frame}

%
\begin{frame}[fragile]{Functors}

In order to be a ``real'' functor, we must also make sure that our instances
obey some laws:

\begin{block}{}
\begin{verbatim}
-- Functor laws:
-- fmap id == id
-- fmap (g . h) == fmap g . fmap h
\end{verbatim}
\end{block}

Note that Haskell can't verify these for us!

\end{frame}

%
\begin{frame}[fragile]{Pointed functors}

It would be nice, given a functor, to be able to take a value and ``inject'' it
into a ``default'' context.

\begin{block}{}
\begin{verbatim}
class Functor f => Pointed f where
  return :: a -> f a
\end{verbatim}
\end{block}

\begin{block}{}
\begin{verbatim}
instance Pointed [] where
  return x = [x]

instance Pointed Maybe where
  return x = Just x
\end{verbatim}
\end{block}

\end{frame}

%
\begin{frame}[fragile]{Bind}

Now we finally get to monads themselves.

\begin{block}{}
\begin{verbatim}
class (Functor m,Pointed m) => Monad m where
  (>>=) :: m a -> (a -> m b) -> m b
\end{verbatim}
\end{block}

We read the operator \texttt{(>>=)} as ``bind.'' The idea is to regard a monad
as an \textbf{action} which produces a value, and bind as a way to sequence two
actions, using the value of the first as the input to the second.

\end{frame}

%
\begin{frame}[fragile]{The \texttt{Maybe} monad}

This is one possible definition of bind for \texttt{Maybe}.

\begin{block}{}
\begin{verbatim}
instance Monad Maybe where
  Just x >>= f = f x
  Nothing >>= _ = Nothing
\end{verbatim}
\end{block}

Recall that we are specializing the type of \texttt{>>=} to

\begin{block}{}
\begin{verbatim}
(>>=) :: Maybe a -> (a -> Maybe b) -> Maybe b
\end{verbatim}
\end{block}

\end{frame}

%
\begin{frame}[fragile]{The \texttt{Maybe} monad}

Bind allows us to write functions like this:

\begin{block}{}
\begin{verbatim}
firstOddAndNextEven :: [Int] -> Maybe (Int,Int)
firstOddAndNextEven lst = 
  find odd lst >>= \x ->
  find (x+1) lst >>= \y ->
  return (x,y)
\end{verbatim}
\end{block}

This looks a little like an imperative program!

\begin{block}{}
\begin{verbatim}
> firstOddAndNextEven [1..10]
Just (1,2)
> firstOddAndNextEven (map (\x -> x * 2 + 1) [1..10])
Nothing
\end{verbatim}
\end{block}

\end{frame}

%
\begin{frame}[fragile]{Do-notation}

In fact, Haskell has some special syntax for monads, so we can write it like
this instead:

\begin{block}{}
\begin{verbatim}
firstOddAndNextEven :: [Int] -> Maybe (Int,Int)
firstOddAndNextEven lst = 
  do x <- find odd lst
     y <- find (x+1) lst
     return (x,y)
\end{verbatim}
\end{block}

Notice the \textbf{do} keyword and the \texttt{<-} operator. This is just
syntactic sugar for the previous definition.

\end{frame}

%
\begin{frame}[fragile]{Monad functions}

There are many library functions which operate on any monad. For example:

\begin{block}{}
\begin{verbatim}
> :t sequence
sequence :: Monad m => [m a] -> m [a]
> sequence ([Just 1, Just 2, Just 3])
Just [1,2,3]
> sequence ([Just 1, Nothing, Just 3])
Nothing
\end{verbatim}
\end{block}

\end{frame}

%
\begin{frame}[fragile]{The \texttt{IO} monad}

In Haskell, IO is a monad.

\begin{block}{}
\begin{verbatim}
countWordsInFile :: FilePath -> IO Int
countWordsInFile f = 
  do h <- openFile f ReadMode
     s <- hGetContents h
     let ws = words s
     return (length ws)
\end{verbatim}
\end{block}

Think of the type \texttt{IO a} as an \emph{IO action which, when run, will
produce a value of type a}. It may also have IO-related side effects.

\end{frame}

%
\begin{frame}[fragile]{The \texttt{IO} monad}

Notice that we can't \emph{run} IO actions in pure code (the type system
prevents it).

\begin{block}{}
\begin{verbatim}
-- will not compile!
untypableFunction :: Int -> Int
untypableFunction n = print n >>= \_ -> (n + 1)
\end{verbatim}
\end{block}

\end{frame}

%
\begin{frame}[fragile]{The \texttt{IO} monad}

So how do we actually \emph{run} an IO action?

\end{frame}

%
\begin{frame}[fragile]{The \texttt{IO} monad}

One option: Haskell could provide a function which would ``run'' the IO action,
and produce its value. Something like this:

\begin{block}{}
\begin{verbatim}
unsafePerformIO :: IO a -> a
\end{verbatim}
\end{block}

In fact, Haskell has such a function. Why is it ``unsafe''?

\end{frame}

%
\begin{frame}[fragile]{The \texttt{IO} monad}

The preferred method is to invoke IO actions from within other IO actions.

In particular, \texttt{main} has type \texttt{IO a}. When we run a program,
\texttt{main} is the root IO action, and it can invoke others.

In \texttt{ghci}, the prompt is ``in'' the IO monad.

\end{frame}

%
\begin{frame}[fragile]{The \texttt{IO} monad}

In Haskell, the IO monad is ``magic,'' in that it the structure used by certain
built-in functions which interact with the operating system.

Note that

\begin{itemize}

\item Other monad instances (e.g. \texttt{Maybe}, lists, many others) are
\textbf{not} magical -- they do what you would expect based on Haskell's usual
semantics and their respective definitions of \texttt{(>>=)}.

\item Other than this single magical aspect, IO is just another monad.

\end{itemize}

\end{frame}

%
\begin{frame}[fragile]{Monadic control flow}

So, IO actions can be stored lazily, passed to pure functions, etc.

\begin{block}{}
\begin{verbatim}
> :t sequence
sequence :: Monad m => [m a] -> m [a]
> let actions = map print [1..3]
actions :: [IO ()]
> let action = sequence actions
action :: IO [()]
> action
1
2
3
[(),(),()]
\end{verbatim}
\end{block}

\end{frame}

%
\begin{frame}[fragile]{Monadic control flow}

We can use this to do some clever control flow.

\begin{block}{}
\begin{verbatim}
> let action' = sequence (reverse actions)
action' :: IO [()]
> action'
3
2
1
[(),(),()]
\end{verbatim}
\end{block}

\end{frame}


